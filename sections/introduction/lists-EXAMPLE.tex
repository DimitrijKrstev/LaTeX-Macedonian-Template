\subsection{Примери за листи}

LaTeX поддржува различни типови на листи за организирање на содржината.

\subsubsection{Ненумерирани листи}

Ненумерираните листи се креираат со околината \texttt{itemize}:

\begin{itemize}
  \item Прва ставка
  \item Втора ставка
  \item Трета ставка
    \begin{itemize}
      \item Вгнездена подставка
      \item Друга подставка
    \end{itemize}
\end{itemize}

\subsubsection{Нумерирани листи}

Нумерираните листи се креираат со околината \texttt{enumerate}:

\begin{enumerate}
  \item Прв чекор
  \item Втор чекор
  \item Трет чекор
    \begin{enumerate}
      \item Подчекор 3.1
      \item Подчекор 3.2
    \end{enumerate}
\end{enumerate}

\subsubsection{Дескриптивни листи}

Дескриптивните листи се креираат со околината \texttt{description}:

\begin{description}
  \item[LaTeX] Систем за подготовка на документи
  \item[Macedonian] Јазик користен во овој шаблон
  \item[UTF-8] Кодирање за текст
\end{description}
